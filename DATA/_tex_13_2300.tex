\begin{tabular}{rrrrrrrrrrrrrrrrrrrr}
\hline
   h1 [mV] &   Delta h1 [mV] &     mu1 [us] &   Delta\_mu1 [us] &   sigma1 [us] &   Delta\_sigma1 [us] &   tau1 [us] &   Delta\_tau1 [us] &     c1 [mV] &   Delta\_c1 [mV] &   h2 [mV] &   Delta h2 [mV] &   mu2 [us] &   Delta\_mu2 [us] &   sigma2 [us] &   Delta\_sigma2 [us] &   tau2 [us] &   Delta\_tau2 [us] &    c2 [mV] &   Delta\_c2 [mV] \\
\hline
  1792.22  &         17.7754 &  0.0436925   &      0.000570567 &     0.0787649 &         0.000763201 &     1.10638 &        0.00345167 &  -1.37442   &       0.0769496 &   358.216 &        10.9297  &    7.89491 &      0.00121618  &     0.0511883 &          0.00149507 &    0.25914  &        0.00428677 &  -2.52194  &       0.0816723 \\
  1352.09  &         18.5095 & -0.00972223  &      0.000698218 &     0.0701123 &         0.00094009  &     1.11646 &        0.00449771 &  -9.07449   &       0.0674968 &   385.527 &         8.73791 &    7.80767 &      0.000935639 &     0.0522194 &          0.00112912 &    0.23395  &        0.00310668 & -11.9968   &       0.0720944 \\
  1848.85  &         14.1214 &  0.0432249   &      0.000432518 &     0.0775555 &         0.000578925 &     1.10134 &        0.00262983 &  -0.0595184 &       0.0600254 &   240.724 &        10.2127  &    7.91566 &      0.00154007  &     0.0473817 &          0.00193378 &    0.281789 &        0.00587105 &  -0.894261 &       0.0649338 \\
  1559.98  &         13.9042 &  0.0218949   &      0.000494593 &     0.0761058 &         0.000663406 &     1.12845 &        0.0030776  &  -3.81357   &       0.0564863 &   474.689 &        13.0838  &    7.84004 &      0.000955118 &     0.0451805 &          0.00119758 &    0.265121 &        0.00361551 &  -5.17046  &       0.0819414 \\
  1759.58  &         16.4733 &  0.0437966   &      0.000535385 &     0.0783202 &         0.000716527 &     1.11182 &        0.00325689 &   3.6201    &       0.0704849 &   443.702 &        12.0977  &    7.89175 &      0.000990179 &     0.0471189 &          0.00123387 &    0.263366 &        0.0036609  &   3.0017   &       0.0803908 \\
  1061.16  &         20.0107 &  0.000989644 &      0.00113276  &     0.0821384 &         0.00151184  &     1.11093 &        0.00672743 &   8.0844    &       0.0916035 &   415.31  &        18.9074  &    7.77721 &      0.00177577  &     0.0506228 &          0.00221269 &    0.283451 &        0.00657622 &   8.35332  &       0.130327  \\
  1683.79  &         13.6942 &  0.0411914   &      0.000505049 &     0.0848592 &         0.000673491 &     1.13646 &        0.00298946 &  -3.25294   &       0.064517  &   363.673 &         9.40276 &    7.82946 &      0.000961763 &     0.0482789 &          0.00119876 &    0.270581 &        0.00356152 &  -4.60998  &       0.0631604 \\
  1777.95  &         16.4103 &  0.035777    &      0.00053944  &     0.0799554 &         0.000720745 &     1.09968 &        0.00322805 & -11.7372    &       0.0728989 &   357.084 &        10.305   &    7.88055 &      0.00112008  &     0.0496011 &          0.00136894 &    0.240997 &        0.00386806 & -14.4845   &       0.0780527 \\
  1710.42  &         15.7768 &  0.0476849   &      0.000560583 &     0.0830129 &         0.000747171 &     1.09441 &        0.00328386 &  -1.19659   &       0.0741535 &   313.736 &        11.4045  &    7.91729 &      0.00137744  &     0.0482017 &          0.00167391 &    0.225313 &        0.00466672 &  -3.85553  &       0.0875072 \\
  1462.26  &         14.9291 &  0.0256149   &      0.000596705 &     0.0799985 &         0.000797889 &     1.11949 &        0.00360592 &  -0.351418  &       0.0654739 &   518.425 &        16.92    &    7.84025 &      0.00106288  &     0.0427424 &          0.00134424 &    0.269607 &        0.00416698 &  -1.55269  &       0.0973303 \\
  1469.53  &         14.2473 &  0.0102827   &      0.000499155 &     0.0707877 &         0.000672104 &     1.13127 &        0.00322429 &   0.269475  &       0.0521289 &   378.702 &        10.0446  &    7.78441 &      0.0010041   &     0.0485133 &          0.0012316  &    0.241467 &        0.00350801 &  -0.46759  &       0.073886  \\
  1655.1   &         19.0329 &  0.0300872   &      0.000642164 &     0.0765648 &         0.000860926 &     1.12418 &        0.00397542 &  -1.66245   &       0.0782077 &   406.434 &        10.399   &    7.83459 &      0.000951615 &     0.0478625 &          0.0011733  &    0.247057 &        0.00338414 &  -3.05251  &       0.0739373 \\
  2078.66  &         20.6704 &  0.0535482   &      0.000558545 &     0.0770207 &         0.000748869 &     1.13419 &        0.00346502 &   4.31574   &       0.0850711 &   353.878 &         8.76387 &    7.96981 &      0.00111289  &     0.0560918 &          0.00132097 &    0.228149 &        0.003527   &   3.43001  &       0.080334  \\
  1819.24  &         16.8039 &  0.0437304   &      0.000539125 &     0.0798515 &         0.00072035  &     1.09877 &        0.00322674 &  -4.62542   &       0.0745577 &   416.115 &        10.6049  &    7.89893 &      0.00104034  &     0.0518414 &          0.00126148 &    0.239337 &        0.0035053  &  -6.28069  &       0.0853087 \\
  1335.69  &         17.8413 &  0.00411771  &      0.000717584 &     0.0737373 &         0.000963607 &     1.11429 &        0.00450351 &  -2.64364   &       0.0699784 &   382.667 &        11.0304  &    7.73939 &      0.00107208  &     0.0478105 &          0.00132003 &    0.244383 &        0.00379425 &  -3.88408  &       0.0789507 \\
  1037.18  &         14.185  & -0.0190979   &      0.000839884 &     0.0841559 &         0.00112465  &     1.23241 &        0.00522897 &   4.50855   &       0.0621545 &   388.771 &         9.62641 &    7.75727 &      0.000884253 &     0.0464159 &          0.00110424 &    0.263749 &        0.00329552 &   4.5799   &       0.0626707 \\
  1175.64  &         15.3511 & -0.0296903   &      0.000583991 &     0.0617283 &         0.000791447 &     1.13574 &        0.0040465  &  -0.0802761 &       0.0460724 &   388.879 &        10.8149  &    7.68767 &      0.00109196  &     0.0504143 &          0.00134246 &    0.255285 &        0.00384857 &  -1.37231  &       0.0801648 \\
  1632.08  &         16.8533 &  0.0357422   &      0.000602403 &     0.079803  &         0.000804793 &     1.09484 &        0.00359929 &  -2.06257   &       0.0749171 &   359.725 &        11.7866  &    7.79912 &      0.00120516  &     0.0473565 &          0.00148703 &    0.245739 &        0.00429635 &  -4.27212  &       0.0830145 \\
   660.342 &         10.7306 & -0.0957357   &      0.00119124  &     0.0998777 &         0.00157993  &     1.25094 &        0.0068707  &  -0.0843175 &       0.0591738 &   354.229 &         9.80386 &    7.68932 &      0.00107519  &     0.0496674 &          0.00131471 &    0.242122 &        0.00371914 &  -1.91946  &       0.0741291 \\
  1669.08  &         15.7095 &  0.0468242   &      0.000578271 &     0.083957  &         0.000771197 &     1.12335 &        0.0034185  &   2.77329   &       0.073551  &   418.002 &        13.6219  &    7.82923 &      0.00116608  &     0.0461725 &          0.00144275 &    0.244258 &        0.00419579 &   1.23033  &       0.0933124 \\
\hline
\end{tabular}